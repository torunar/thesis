\section*{Аннотация}

\noindent \textbf{Дипломный проект} Щекотова Михаила Михайловича по теме <<САПР для разработки программного обеспечения программируемых логических интегральных схем>>.
Руководитель Грачева Наталья Олеговна.
Защищён на кафедре <<Измерительно-вычислительные комплексы>> УлГТУ в 2014 году.\\*[0.2cm]

\noindent \textbf{Пояснительная записка:} 120 с., \total{section} разд., \total{appendix} прил., \total{figure} рис., \total{table} табл., 12 ист.\\*[0.2cm]

\noindent \textbf{Ключевые слова:} САПР, ПЛИС, VHDL, Qt, свободное ПО.\\*[0.2cm]

\noindent САПР предназначена для разработки программного обеспечения ПЛИС.

\noindent Главной особенностью системы является расширяемость за счет пользовательских модулей элементов.

\noindent Разработка схем ведется в режиме визуального проектирования.

\noindent Поддерживается экспорт в формате графического файла и программного кода.

\noindent Система состоит из 4 подсистем: <<Окно программы>>, <<Окно документа>>, <<Документ>> и <<Элемент>>, главной из которых является <<Документ>>.

\noindent САПР реализована на языке C++ с использованием фреймворка Qt.

\noindent Является свободным кроссплатформенным программным обеспечением.