\subsection{Организация информационного обеспечения}

В разрабатываемой в рамках дипломного проекта системе принято решение об организации доступа к данным на уровне отдельных файлов и отказа от использования СУБД.

Данный подход принят из соображения, что разрабатываемая система является прикладным приложением для персональных компьютеров, ориентированным на индивидуальное использование.

Ввод данных осуществляется в автоматизированном режиме: используя графические элементы интерфейса и расширения системы, пользователь имеет возможность создавать программные решения на VHDL, описывающие работу соответствующих устройств.

Ввод и вывод данных в системе ведутся с использованием файлов:
\begin{itemize}
  \item файлы документов;
  \item файлы расширений;
  \item графические файлы;
  \item файлы исходного кода;
  \item файл настроек.
\end{itemize}

Как файлы документов, так и файлы расширений являются файлами формата XML специализированного вида.
XML выбран в качестве формата по следующим причинам:
\begin{itemize}
  \item семантическое соответствие~--- представление схемы в виде элементов со свойствами идеально соответствует XML-дереву (tree) с узлами (node), обладающими атрибуами (attribute);
  \item раздельное хранение структуры и данных~--- позволяет осуществлять обработку документа стандартизированными методами, не привязываясь к структуре файла;
  \item наличие полноценной поддержки работы с XML-деревьями в Qt;
\end{itemize}

Экспорт результатов работы программы может быть осуществлен в графический файл в формате .PNG.
PNG~--- растровый формат хранения графической информации, использующий сжатие без потерь по алгоритму Deflate.

Файлы исходного кода содержат описание структуры и функционирования схемы в соответствие со стандартом VHDL93.

Хранение настроек программного обеспечения осуществляется, в зависимости от операционной системы, либо с помощью файлов (Unix-подобные системы, включая Mac OS), либо с помощью системного реестра (ОС Microsoft Windows).
Независимо от операционной системы настройки хранятся в виде множества пар <<ключ -- значение>>, разбитых по категориям.

Подобный подход к организации информационного обеспечения позволяет обеспечить соответствие следующим требованиям:
\begin{itemize}
  \item Переносимость решений.\\
  Файлы, созданные в SVE на одной персональной ЭВМ, могут быть перенесены на другую.
  \item Легкость распространения расширений.\\
  Поскольку расширения также являются файлами, возможен обмен ими как путем прямого обмена файлами между пользователями, так и на уровне организации централизованных файловых хранилищ с публичным или же ограниченным доступом.
  \item Простота резервного копирования и архивации.\\
  В случае необходимости создания резервной копии рабочих файлов или же системы в целом, этот процесс можно осуществить путем выполнения простых операций на файловой системе, механизм осуществления которых присутствует в любой из целевых ОС.
\end{itemize}


\subsection{Проектирование файлов данных}

Ввод и вывод данных в системе ведется с использованием файлов.

Файлы проектов, содержащие проектируемую пользователем схему, имеют расширение *.SVE.
Файл .SVE~--- основной файл документов <<SVE>>.
Он содержит в себе описания рабочего пространства и представления схемы.
Представляет из себя XML-файл с описанием узлов схемы и связей между ними;

Подключаемые расширения располагаются на уровне ФС в папках, каждая из которых содержит следующие файлы:
\begin{itemize}
  \item файл в формате *.SVX, задающий функциональную и структурную сущность узла;
  \item векторное изображение в формате *.SVG, используемое как пиктограмма на схеме.
  Использование векторного изображения позволяет с легкостью осуществлять масштабирование без потерь качества и, следовательно, использовать указанное изображение как в схеме, так и в качестве пиктограммы на панели инструментов;
\end{itemize}

Файл .SVX~--- файл расширения.
Описывает подключаемое расширение.
Представляет собой описание вентиля или целой сложной подсистемы, являющейся частью данной схемы.

В файле расширения задаются:
\begin{itemize}
  \item название;
  \item описание;
  \item автор;
  \item структура узла.
\end{itemize}

Файлы *.SVE и *.SVX явлются XML-документами с указанием специализированного типа документа (doctype).

Рассмотрим структуру файла .SVE на примере:
\begin{lstlisting}
<!DOCTYPE SVE>
<document height="560" width="980">
  <node id="1394441231792" x="10" y="10" plugin="and_gate"/>
  <node id="1394441236477" x="10" y="80" plugin="not_gate"/>
  <node id="1394441238420" x="10" y="150" plugin="or_gate"/>
  <label id="1394441242496" x="100" y="10" text="AND"/>
  <label id="1394441248257" x="100" y="80" text="NOT"/>
  <label id="1394441256245" x="100" y="150" text="OR"/>
  <link first_id="1394441231792" last_id="1394441236477" id="1394441465274"/>
</document>
\end{lstlisting}

Указанный файл описывает документ SVE~--- на это указывает первая строка, <!DOCTYPE SVE>.

Рабочая область документа (элемент <document>) имеет размер 980x560 точек (атрибуты <<width>> и <<height>>).

В рабочей области расположены три узла~--- элементы <node>, соответствующие расширениям AND, NOT и OR (атрибут <<plugin>>), имеющие уникальные идентификаторы (атрибут <<id>>) и расположенные в координатах (10;10), (10;80) и (10;150), соответственно (атрибуты <<x>> и <<y>>).

Узлы AND (id="1394441231792") и NOT (id="1394441236477") соединены с помощью связи~--- элемент <link>, атрибуты <<first\_id>> и <<last\_id>> которого задают источник и приемник связи.

Рассмотрим файл расширения (*.SVX):

Пример валидного файла .SVX:

\begin{lstlisting}
<!DOCTYPE SVX>
<plugin name="or_gate">
  <info author="Mike Schekotov" description="Core OR gate"/>
  <in  name="IN\%INC\%"  />
  <in  name="IN\%INC\%"  />
  <out name="OUT\%OUTC\%"/>
  <src><![CDATA[\%OUT_1\% <= \%IN_1\% or \%IN_2\%;]]></src>
</plugin>
\end{lstlisting}

Указанный файл описывает расширение~--- на это указывает первая строка, <!DOCTYPE SVX>.

Расширение (элемент <plugin>) носит название <<or\_gate>>.

В информации о расширении (элемент <info>) указаны автор (атрибут <<author>>) и описание (атрибут <<description>>).

Описываемый узел имеет два входа (элементы <in>) и один выход (элемент <out>).
При именовании входов и выходов используются подстановочные шаблоны \%INC\% (<<In Counter>>~--- счетчик входов) и \%OUTC\% (<<Out Counter>>~--- счетчик выходов).
В процессе работы по мере добавления узлов в документ эти счетчики увеличиваются, что позволяет обеспечить уникальные автоимена входов и выходов.

Функционирование узла описывается элементом <src>, в котором в секции <![CDATA]]> задается исходный код.
При написании кода также используются подстановочные шаблоны: \%IN\_*\% и \%OUT\_*\%, в которых используются порядковые индексы входов и выходов.
Использование подобных шаблонов позволяет обеспечить универсальный механизм адресации к компонентам узла.

Время загрузки одного SVE-файла пропорционально его размеру и для файлов менее 1 МБ не превышает 20 секунд.
Время подключения одного SVX-файла и его загрузка для использования в программу для модулей менее 1 МБ не превышает 20 секунд.

Настройки системы хранятся в виде множества <<ключ~--- значение>>.
В зависимости от операционной системы, на которой выполняется программа, эти настройки хранятся:
\begin{itemize}
  \item В Microsoft Windows: в системном реестре по адресу HKEY\_LOCAL\_MACHINE\textbackslash mike-schekotov\textbackslash sve\textbackslash\\
  Опции представляют собой отдельные ключи в реестре.
  \item В ОС на базе ядра Linux: /home/user/.config/mike-schekotov/sve.conf, где <<user>>~--- имя текущего пользователя.\\
  Конфигурационный файл является простым текстовым файлом с очевидной стуктурой, где каждая опция задается отдельной строкой.
  \item В Mac OS: /Users/user/.config/mike-schekotov/sve.conf, где <<user>>~--- имя текущего пользователя.
\end{itemize}

Пример файла настроек:

\begin{lstlisting}
[default_doc]
blank_size=@Size(980 560)
node_size=@Size(80 60)
[plugins]
enabled=or_gate, not_gate, and_gate
icon_size=@Size(32 32)
on_panel=or_gate, not_gate, and_gate
plugin_dir=../plugins/
\end{lstlisting}

Указанные в квадратных скобках выражения указывают секцию, к которой относятся последующие настройки.
В Microsoft Windows отдельной секции соответствует вложенная относительно указанного пути ветка дерева реестра.

Вложенность опций в категорию распространяется на все опции, идущие до объявления следующей секции.
Название опции отделено от значения знаком равенства.

Поскольку запись настроек ведется через унифицированные механизмы класса QSettings, в конфигурационном файле используются некоторые специфичные обозначения, так, например, запись вида <<@Size(X Y)>> показывает, что при чтении данной опции значение должно быть преобразовано к типу QSize, задающему размер в виде пары \{ширина; высота\}.

В том случае, если опция хранит несколько значений, они разделяются запятой и при чтении преобразуются к типу QList, описывающему список элементов указанного типа.

Для экспорта система использует следующие виды файлов:
\begin{itemize}
  \item Файл .VHDL~--- файл, содержащий программный код на языке VHDL.
  \item Файл .PNG~--- изображение проектируемой схемы в графическом формате.
\end{itemize}