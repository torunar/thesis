В рамках данной пояснительной записки к дипломному проекту раскрываются аспекты создания системы автоматизированного проектирования программного обеспечения ПЛИС, получившей название <<SVE>>~--- <<Simple VHDL Editor>>.

Объектом данного проекта является проектирование программного обеспечения ПЛИС. Предмет работы составляет разработка системы автоматизированного проектирования.

Проект разрабатывается в интересах дальнейшего внедрения на кафедре <<Измерительно-вычислительные комплексы>> Ульяновского государственного технического университета в качестве инструмента моделирования ИС.
В настоящее время на кафедре имеется ряд ПЛИС компании <<Altera>>, слабо интегрированных в учебный процесс в виду отсутствия простой системы, предназначенной для разработки ПО ПЛИС.
Менее половины имеющихся ПЛИС задействованы в лабораторном практикуме, при этом разработка ИС на них затруднена.

Цели разработки и внедрения проекта:
\begin{itemize}
  \item Повышение самостоятельности работы студентов и сокращение нагрузки на преподавателей.\\
  Поскольку разработка ПЛИС на базе аппаратно-технического комплекса~--- достаточно гибкий процесс, обеспечивающий определенную устойчивость к ошибкам проектирования со стороны учащихся, отсутствует необходимость в наличии постоянного контроля со стороны преподавателя.
  \item Сокращение затрат на натурное моделирование и испытания.\\
  Возможность построения широкого спектра систем на базе ПЛИС позволяет отказаться от использования реальных приборов и проведения их технического обслуживания (диагностики, ремонта).
  \item Возможность получения дистанционного образования.\\
  Существование ПЛИС делает возможным создание виртуальных лабораторий на основе виртуальных приборов,  построенных на базе ПЛИС.
\end{itemize}

Задачи, выполнение которых способствует достижению поставленных целей:
\begin{itemize}
  \item Создание материально-технической базы.
  \item Создание и внедрение дидактических пособий и руководств по разработке ИС на базе ПЛИС.
  \item Обеспечение наглядности и понятности процесса проектирования.
  \item Обеспечение повторяемости и воспроизводимости создаваемых проектных решений или их частей.
\end{itemize}

В первом разделе пояснительной записки к дипломному проекту приводится техническое задание на разрабатываемую систему.
Здесь рассматривается процесс проектирования программного обеспечения для ПЛИС, формулируются основные требования к функциям разрабатываемой системы и к видам ее обеспечения, анализируется актуальность проводимой разработки.

Раздел \ref{sec:funcprot} посвящен информационному моделированию разрабатываемой системы и содержит описание ряда процессов, протекающих в ней.

Раздел \ref{sec:concurents} посвящен подробному анализу конкурентных разработок~--- САПР, созданных другими компаниями. Здесь проводится оценка их применимости в контексте учебного процесса, подробно анализируются их возможности, на основе чего формулируются функциональные требования к разрабатываемой системе.

В разделе \ref{sec:informational-supply} описываются структура программного обеспечения, функции его компонентов, обосновывается выбор компонентов ПО, приводится описание и особенности процесса разработки прикладного программного обеспечения, эксплуатации и сопровождения системы.
Также в данном разделе описан пользовательский интерфейс разработанной системы и приведено руководство пользователя.

Раздел \ref{sec:mathematical-supply} содержит описание ряда алгоритмов, разработанных в рамках создания дипломного проекта.

Тестированию системы в различных режимах проекта посвящен раздел \ref{sec:testing}.

Разделы \ref{sec:economics} и \ref{sec:safety} содержат в себе аналитическую оценку экономической целесообразности проекта и оценку безопасности и экологичности проекта.

В процессе работы над проектом был использован ряд пособий российских и зарубежных авторов, а также справочная информация по языкам программирования.

В работе <<Цифровая схемотехника>> Угрюмова Е. П. излагаются многие аспекты, связанные с использованием, изучением, разработкой и проектированием цифровых элементов, выполняющих задачи обработки информации, приводятся методики проектирования узлов и схем.
В главе 8 приводятся вводные сведения о СБИС с программируемой логикой и FPGA- и CPLD-системах, границах их применения, особенностях функционирования и популярных семействах ПЛИС.

Пособия <<Проектирование на ПЛИС. Курс молодого бойца>>, <<Проектирование цифровых устройств с использованием ПЛИС>>, <<Проектирование цифровых устройств на программируемых логических интегральных схемах>> и статья <<ПЛИС>> посвящены единой тематике~--- разработке систем на базе ПЛИС.

Работа <<Проектирование на ПЛИС. Курс молодого бойца>> Максфилда К., содержит обзор и анализ схемотехнических подходов к проектированию ПЛИС и методов проектирования с использованием языков описания аппаратных средств.
В книге приводится ряд уникальных сведений о методах проектирования с использованием различных языков программирования и виртуального макетирования.

В <<Проектировании цифровых устройств с использованием ПЛИС>>, помимо вводных данных о программируемых логических схемах, описана последовательность проектирования цифровых устройств с использованием ПЛИС.

В труде <<Проектирование цифровых устройств на программируемых логических интегральных схемах>> Бутаева М. М. и коллектива авторов уделяется внимание проектированию цифровых устройств на ПЛИС с применением ЭВМ и способах непосредственного программирования устройств.
Приведены основные этапы разработки ПЛИС, раскрыта главенствующая роль моделирования в процессе проектирования ПЛИС.

Пособие <<Основы языка VHDL>> Бибило П. Н. содержит описание языка описания аппаратуры интегральных схем VHDL и основных принципов представления интегральных схем с использованием VHDL.

Для анализа возможностей конкурентных САПР использовалась документация к <<Altera Quartus II>> и <<Xilinx ISE>>

Работа <<Основы САПР>> Ли К. содержит сведения о принципах построения и разработки САПР.

В процессе создания программного обеспечения активно использовалась документация по кроссплатформенному инструментарию разработки Qt, изложенная в справочных статьях <<All Classes>> и <<Qt Creator 3.1>>.