\subsection{Назначение и цели создания системы} \label{sec:purpouse}
\subsubsection{Назначение системы} \label{sec:purpouse:misson}
Система предназначена для автоматизации проектной деятельности при разработке функциональных схем виртуальных цифровых устройств за счет визуализации процесса моделирования с использованием языка описания аппаратуры интегральных схем VHDL.

Возможным направлением внедрения системы является интеграция ее в учебный процесс в высших учебных заведениях технического профиля для приобретения практических навыков по дисциплинам <<Цифровая электроника>>, <<Электротехника и электроника>>, <<Цифровые вычислительные устройства и микропроцессорные системы>> и иным схожего профиля.
%
%
\subsubsection{Цели создания системы} \label{sec:purpouse:objectives}
В результате внедрения системы будут достигнуты следующие технологические показатели:
\begin{itemize}
  \item сокращение сроков проектирования интегральных схем;
  \item сокращение трудоемкости проектирования;
  \item сокращение затрат на натурное моделирование и испытания.
\end{itemize}

А также, ряд производственно-экономических показателей:
\begin{itemize}
  \item сокращение затрат на приобретение специализированного программного обеспечения;
  \item сокращение затрат на обслуживание и ремонт аппаратно-технических комплексов учебных стендов в виду замены их на виртуальные стенды.
\end{itemize}
%
%
%
\subsection{Характеристика объекта автоматизации} \label{sec:characteristics}
Объектом автоматизации является процесс проектирования функциональной схемы, использующейся при разработке программного обеспечения для программируемых логических схем.

\subsubsection{Общее описание} \label{sec:characteristics:summary}
ПЛИС является гибким и многофункциональным устройством, идеально подходящим для использования в учебной деятельности для получения практических навыков разработки программного обеспечения для встраиваемых систем.
В отличие от обычных цифровых схем, логика работы ПЛИС не определяется при изготовлении, а задается посредством программирования.
ПЛИС широко используются для построения различных по сложности и по возможностям цифровых устройств.
Внедрение ПЛИС в процесс обучения позволяет, во-первых, способствовать повышению профессионализма выпускников за счет обучения основам работы с цифровыми системами, во-вторых, отказаться от использования механически изношенных и устаревших морально аппаратных комплексов, использующихся на сегодняшний день во многих учебных заведениях.

Внедрение ПЛИС в учебный или производственный процесс сопровождается внедрением САПР.
САПР представляет из себя приложение, задача которого~--- упрощение и ускорение процесса разработки программного обеспечения для ПЛИС.
САПР позволяет вести большую часть разработки на визуальном уровне, сокращая как технологические, так и экономические затраты на написание и отладку программного кода.
Использование САПР также делает возможным повторное использование частей ранее созданных проектных решений.

Проектирование интегральных схем в САПР ведется на уровне структурного описания: схема составляется из готовых узлов, что позволяет достичь, с одной стороны, гибкости процесса разработки, а с другой~--- достаточной его простоты.
Узлы представлены как простейшими логическими функциями~--- вентилями, так и более сложными компонентами, являющимися целыми подсистемами.

Для представления готовой схемы используется язык VHDL, предназначенный для программирования логических интегральных схем.
VHDL используется многими разработчиками при написании программного обеспечения для ПЛИС, так как позволяет описать не только схему как с точки зрения структуры, так и с точки зрения и ее функционирования.
Код на VHDL, будучи скомпилированным, может быть загружен в ПЗУ ПЛИС.
VHDL является независимым от архитектуры устройства решением~--- его использование позволяет отказаться от написания программы, ориентированной на работу на конкретном устройстве в пользу создания универсального решения, которое может быть развернуто на любом устройстве после внесения незначительных изменений, обусловленных аппаратным конструктивом конкретной ПЛИС.
%
%
\subsubsection{Структура и принципы функционирования} \label{sec:characteristics:structure}

Программируемые логические интегральные схемы представляют собой цифровые интегральные схемы, состоящие из программируемых логических блоков и программируемых соединений между этими блоками \cite[c.~18]{maxfield}.
Класс устройств ПЛИС представляет из себя развитие таких цифровых машин, как элементы ПМЛ, ПЛМ и БМК.
Устройства класса ПЛИС принадлежат к классу сверхбольших интегральных схем.

Иногда под термином <<ПЛИС>> понимают не только аппаратно-технический комплекс, но и непосредственно запрограммированную на его базе программную логическую интегральную схему \cite[c.~16]{popov}.
Программируемые логические интегральные схемы~--- это интегральные схемы с сильной интеграцией с регулярной (постоянной) структурой, которые могут программироваться пользователем для выполнения заданной функции.
В зависимости от устройства, ПЛИС могут программироваться либо один раз, либо быть многократно программируемыми.
Многократно программируемые ПЛИС носят название интегральных схем с гибкой логикой или FLEX-систем \cite[c.~15]{butaev}.

Современные высокотехнологичные ПЛИС, помимо матрицы программируемых элементов, содержат собственные модули оперативной памяти, процессорные ядра и высокоскоростные модули обработки ввода-вывода, называемые аппаратными ядрами.
Подобное расширение аппаратной части ПЛИС привело к возникновению и развитию концепции ПЛИС-платформы.
Суть ее заключается в том, что при необходимости построения системы на базе ПЛИС разработчик может воспользоваться уже спроектированными ПЛИС-платформами для собственных изделий \cite[c.~425]{ugryumov}.

Несмотря на то, что сама концепция ПЛИС подразумевает построение максимально гибкой системы за счет использования как программируемых логических блоков, так и программируемых связей между ними, иногда разработчики прибегают к построению систем, в которые интегрируются заказные микросхемы с жестко заданной функциональностью.
В этом случае процесс производства и проектирования ПЛИС упрощается, несмотря на определенную потерю гибкости \cite[c.~57-58]{maxfield}.

Сфера применения ПЛИС крайне широка: на них могут строиться как крупные блоки (подсистемы) систем, так и системы в целом, включая память и процессорные устройства.

Часто ПЛИС используются при отработке прототипов систем при их проектировании, даже если конечная реализация рассчитана на применение иных аппаратных компонентов, и при создании малотиражных изделий быстрыми и эффективными методами \cite[с.~392]{ugryumov}.
Подобная область применения объясняется тем, что само появление электронных схем с программируемой логикой функционирования изначально было вызвано потребностью в нестандартных компонентах, выпуск которых в виде заказных ИС в большинстве случаев экономически не целесообразен \cite[c.~4]{popov}.

Однако, основное применение ПЛИС находят в построении динамически реконфигурируемых устройств: так, например, задачи кодирования и декодирования при обработке данных никогда не выполняются одновременно для одного и того же набора данных, а значит, можно иметь одну ПЛИС с двумя разными программами, хранящимися в ОЗУ, осуществляя переключение между ними по мере необходимости \cite[c.~410]{ugryumov}.

К преимуществам современных ПЛИС можно отнести \cite[c.~17]{popov}:
\begin{itemize}
  \item Простоту и малое время, затраченное на проектирование.
  \item Низкую стоимость разработки.
  \item Сокращение используемого пространства печатных плат за счет использования однотипных компонентов и сильной интеграции на кристалле.
  \item Более низкую стоимость в расчете на одну микросхему по сравнению с заказными ИС.
  \item Более продолжительное время обращения продукта на рынке благодаря возможности перепрограммирования.
  \item Возможность создания динамически реконфигурируемых устройств.
\end{itemize}

Однако, имеются у ПЛИС и недостатки: более низкая скорость работы по сравнению с полностью заказными интегральными схемами, а также нерентабельность использования в крупносерийном производстве \cite[c.~447]{ugryumov}.

При проектировании ПЛИС используют методы, аналогичные используемым при создании классических микросхем (вентильных матриц, микропроцессорных систем на стандартных элементах).
Основные компоненты ПЛИС~--- программируемые логические блоки и соединения~--- создаются с учетом требований компактности размещения на поверхности кристалла.
Вспомогательные блоки ОЗУ, ввода-вывода, дополнительных обработчиков, как правило, не имеют жестких ограничений по размещению, поэтому изготавливаются они по методам создания схем на стандартных элементах \cite[c.~59]{maxfield}.

Для программирования используются ПЛИС используются аппаратные устройства специального класса~--- программаторы и отладочные среды, позволяющие задать желаемую структуру цифрового устройства в виде принципиальной электрической схемы или программы на специальных языках описания аппаратуры \cite[c.~18]{popov}.

Рассматривая ПЛИС и моделирование с их применением в рамках как производственного, так и учебного процесса, можно выделить ряд задач, решаемых на устройствах данного класса:
\begin{itemize}
  \item Высокоскоростная обработка данных и алгоритмы цифровой обработки сигналов.
  \item Модельные стенды.
  \item Виртуальные лаборатории.
  \item Системы дистанционного обучения, построенные на базе виртуальных лабораторий.
  \item Системы логической эмуляции, реализующие основные особенности поведения разрабатываемых систем на промежуточном этапе проектирования ПЛИС.
\end{itemize}

Использование ПЛИС как средства эмуляции и моделирования не отменяет классических методов как разработки и трассировки схем, так и программного моделирования их работы, но дополняет их, позволяя сократить затраты на первые и повысить точность представления вторых \cite[c.~411]{ugryumov}.


\subsubsection{Существующая информационная система и ее недостатки} \label{sec:characteristics:analysis}
В настоящее время используемыми системами являются продукты Quartus II компании Altera и Xilinx ISE компании Xilinx.
Эти системы обладают рядом схожих недостатков:
\begin{itemize}
  \item Сложность в освоении.\\
  Поскольку данные САПР расчитаны, в первую очередь, на профессиональное использование, их применение в учебном процессе затрудняется значительной сложностью интерфейса.
  Являясь программными комплексами, ориентированными на полноценную разработку ПО для ПЛИС, упомянутые продукты предоставляют значительную функциональность, избыточную в контексте учебного процесса.
  \item Тяжеловесность.\\
  Из соображений надежности и кроссплатформенности Altera Quartus II и Xilinx ISE выполнены на языке Java, что обуславливает необходимость дополнительной прослойки между ОС и САПР в виде среды исполнения Java (JRE).
  К тому же, широкий спектр решаемых данными САПР задач сам по себе объясняет их значительное ресурсопотребление.
\end{itemize}

Указанные критерии делают данные САПР малопригодными для использования в обучении.
%
%
\subsubsection{Анализ аналогичных разработок} \label{sec:characteristics:analogue}
В качестве аналогов рассмотрим продукты Altera Quartus II и Xilinx ISE.

Выбор именно этих продуктов для сравнения обусловлен рядом факторов:
\begin{itemize}
  \item Распространенность и доступность ПЛИС от компаний Altera.\\
  ПЛИС <<Cyclone>>, в частности, успешно используются в составе многих виртуальных лабораторных стендов;
  \item Условные бесплатность указанных программных решений.\\
  И Altera Quartus II, и Xilinx ISE являются бесплатными для персонального использования, но имеют при этом ограниченные возможности.
  \item Доминирующее положение на рынке.\\
  Altera Quartus II и Xilinx ISE~--- крупные программные пакеты, готовые для решения множества задач: программный синтез, временной анализ, визуальное проектирование и ряд других.
  \item Задействованность их на настоящий момент в учебном процессе.
\end{itemize}

\small
\singlespacing
\begin{longtable}[h]{|p{0.12\textwidth}|p{0.2\textwidth}|p{0.29\textwidth}|p{0.29\textwidth}|}
  \caption{Сравнение системы и аналогичных разработок}
  \label{table:cad-compare}
	\\ \hline
	  \textbf{Критерий}                                              &
	  \textbf{SVE}                                                   &
	  \textbf{Altera Quartus} II                                     &
	  \textbf{Xilinx ISE}
	\\ \hline
  \endfirsthead

  \multicolumn{4}{l}{\normalsize Продолжение таблицы \thetable{} \small}
  \\ \hline
    \textbf{Критерий}                                              &
    \textbf{SVE}                                                   &
    \textbf{Altera Quartus} II                                     &
    \textbf{Xilinx ISE}
    \\ \hline
  \endhead

	  Основные возможности                                            &
	  $\bullet$ Проектирование логических схем функционирования ПЛИС. \newline
	  $\bullet$ Создание программного описания функционирования схем. &
	  $\bullet$ Проектирование логических схем функционирования ПЛИС. \newline
	  $\bullet$ Создание программных описаний функционирования схем.  \newline
	  $\bullet$ Временной анализ.                                     \newline
	  $\bullet$ Загрузка программы в ПЗУ ПЛИС.                        &
	  $\bullet$ Проектирование логических схем функционирования ПЛИС. \newline
	  $\bullet$ Создание программных описаний функционирования схем.  \newline
	  $\bullet$ Временной анализ.                                     \newline
	  $\bullet$ Загрузка программы в ПЗУ ПЛИС.
  \\ \hline

	  Целевые ПЛИС                                                    &
	 ~---                                                             &
	  Семейства Cyclone, Arria, Stratix                               &
	  Семейства Virtex, Spartan, Coolrunner, XC9500
  \\ \hline

	  Язык программирования                                           &
	  C++                                                             &
	  Java                                                            &
	  Java
  \\ \hline

	  Рабочие платформы                                               &
	  Любой Linux-дистрибутив, FreeBSD, Microsoft Windows             &
	  Red Hat Enterprise Linux,
	  Suse Linux Enterprise, FreeBSD, Microsoft Windows               &
	  Red Hat Enterprise Linux, Suse Linux Enterprise,
	  Solaris, Microsoft Windows
  \\ \hline

	  Вид ПО                                                          &
	  Свободное, LGPL                                                 &
	  Проприетарное                                                   &
	  Проприетарное
  \\ \hline

	  Стоимость одной лицензии                                        &
	  Бесплатное распространение                                      &
	  Бесплатное распространение (Web Edition),                       \newline
	    2995\$ (Стандартная подписка)                                 &
	  Бесплатное распространение (Web Edition),                       \newline
	    2995\$ (Стандартная подписка)
  \\ \hline

	  Языки синтеза                                                   &
	  VHDL                                                            &
	  VHDL, AHDL, Verilog                                             &
	  VHDL, Verilog
  \\ \hline

	  Размер установленной системы                                    &
	  менее 100 Мб                                                    &
	  5.7 ГБ                                                          &
	  4.2 ГБ
  \\ \hline
\end{longtable}
\normalsize
\onehalfspacing

На основе приведенного в таблице \ref{table:cad-compare} сравнения становится очевидно, что разрабатываемая система, во-первых, предназначена для работы в условиях ограниченных ресурсов, во-вторых, ставит свой целью охватить только конкретные учебные задачи.
C++, как язык разработки, обуславливает высокую производительность приложения, исключая прослойку в виде Java-машины, а модель лицензирования позволяет свободно использовать ее в учебном процессе, не нарушая условия лицензирования.

Помимо этого, конкурентные аналоги являются профессиональными продуктами, не предназначенными для работы непрофессионалов.
Сложность их освоения достаточно велика, что обуславливается как непривычным пользовательским интерфейсом, так и общей развитостью и масштабностью указанных САПР.

Таким образом, САПР <<SVE>> является более подходящей для использования в учебном процессе.
%
%
\subsubsection{Актуальность проводимой разработки} \label{sec:characteristics:actual}
Поскольку в настоящее время наблюдается тенденция использования ПЛИС и виртуальных стендов в ряде дисциплин ВУЗов, имеется необходимость в САПР, достаточно простой для освоения, не требовательной к ресурсам пользовательской ЭВМ, готовой к быстрому разворачиванию и интеграции в процесс обучения, работающей единообразно под различными операционными системами и предоставляющей универсальные методы работы с ПЛИС, можно утверждать, что проводимая разработка является актуальной.
Предъявленные требования обуславливаются как учебной программой подготовки специалистов, так и материальной обеспеченностью учебных заведений.

Выполнение означенных требований обусловлено следующими особенностями проектирования и разработки:
\begin{itemize}
  \item простота: пользовательский интерфейс спроектирован максимально простым и приближенным к виду простых схемных и графических редакторов;
  \item быстрота внедрения: система, будучи разработанной как независимое приложение, готова к внедрению на ПЭВМ, удовлетворяющих системным требованиям~--- без необходимости установки дополнительных программных пакетов;
  \item легковесность и платформонезависимость: за счет использования Qt в качестве основного инструментария разработки становится возможным создать кросс-платформенное решение, обладающее необходимыми показателями производительности и надежности;
  \item универсальность: полученный в результате работы САПР программный код на VHDL может быть применен впоследствии в прочих специализированных программных продуктах: визуализаторах, симуляторах работы интегральных схем, а также преобразован для непосредственной загрузки в ПЗУ ПЛИС.
\end{itemize}
%
%
%
\subsection{Общие требования к системе} \label{sec:requirements}

\subsubsection{Требования к структуре и функционированию системы} \label{sec:requirements:system}
Система должна быть функционально разделена на уровни, находящихся во взаимодействии:
\begin{itemize}
  \item Уровень представления (back-end).\\
  На этом уровне система должна осуществлять непосредственную работу: импорт расширений, файловые операции.
  \item Уровень взаимодействия (front-end).\\
  На данном уровне система должна выполнять работу с точки зрения пользователя: визуальное проектирование схем, вывод результатов преобразований, операции со схемой.
\end{itemize}

Автоматизации подлежат следующие операции:
\begin{itemize}
  \item сохранение и загрузка файлов;
  \item преобразование схемы в программный код на VHDL;
  \item экспортирование схемы в формат PNG;
  \item загрузка пользовательских модулей.
\end{itemize}

Система должна обеспечивать сохранение и загрузку файлов, поскольку без реализации указанных операций разработка системы лишена смысла: создание программного SVE-файла вручную представляется задачей если не невозможной, то крайне затратной с точки зрения времени и трудовых затрат пользователя, а отсутствие загрузки ранее созданных решений противоречит ранее заявленной концепции САПР, как таковой.

Система дожна предоставлять средства автоматизации процесса создания программного кода, т.к. это процесс также является сложным и затратным по времени, при осуществлении вручную.
Автоматизации должны подвергаться фазы создания описания сущностей схемы, алгоритмов их функционирования и работы системы в целом, позволяя существенно сокращать время на отладку и тестирование полученного кода, уменьшив влияние человеческого фактора в процессе создания программного кода.

Система должна предоставлять возможность экспорта схемы в формат изображения для дальнейшего его использования при составлении отчетов либо передачи третьим лицам, не имеющим установленной системы <<SVE>>.

Система должна обеспечивать расширяемость за счет пользовательских модулей, что позволяет обеспечить пользователя необходимыми инструментами для работы~--- модули ядра включают лишь базовые вентили, не обеспечивая требуемой гибкости при проектировании сложных систем.
Пользовательские расширения же, напротив, обеспечивают необходимое наращивание функциональности.
Будучи импортированным в программу, расширение должно становится полноценным инструментом~--- именно для обеспечения пользователя развитым инструментарием процесс импорта должен быть автоматизирован.

Поскольку особый упор в разработке системы делается на подключаемые модульно пользовательские расширения, система имеет значительный потенциал: наращивая количество и сложность модулей, возможно добиться, с одной стороны, необходимой степени повторного использования успешных проектных решений, с другой стороны~--- простоты и скорости процесса разработки.
%
%
\subsubsection{Дополнительные требования} \label{sec:requirements:additional}
Система должна быть исполнена так, чтобы ее обслуживание на всех стадиях, от установки до использования, возможно было провести силами одного человека~--- конечного пользователя.

Пользовательский интерфейс должен быть спроектирован так, чтобы обеспечить комфортный доступ к возможностям САПР.
Элементы управления и иные компоненты интерфейса должны быть легко различимы и не должны создавать избыточной нагрузки на зрительную систему пользователя.

Необходимо обеспечить соответствие следующим стандартам:
\begin{itemize}
  \item стандарт языка C++ ISO/IEC 14882:2011;
  \item стандарты языка описания аппаратуры цифровых систем VHDL ГОСТ Р 50754-95 и IEEE Std 1076-2008;
  \item стандарт W3C для языка разметки XML 1.1;
  \item стандарт POSIX.1-2001 для архивных файлов tar.
\end{itemize}

Условия использования компонентов, входящих в состав системы, должны быть согласованы с лицензионным соглашением LGPL, что делает возможным их распространение под любой лицензией, несовместимой с GNU/GPL. Это позволяет осуществлять распространение системы как формате коммерческого проприетарного, так и в виде бесплатного приложения с открытым программным кодом.
%
%
%
\subsection{Требования к функциям, выполняемым системой} \label{sec:functions}

\subsubsection{Сохранение и загрузка файлов схем}
В системе должна быть реализована возможность использования ранее созданных проектных решений.
Для этого необходима возможность сохранить результат работы в совместимом с системой формате.

Процесс сохранения должен переносить все аспекты проектируемой схемы, как то:
\begin{itemize}
  \item используемые в схеме узлы (список модулей, из которых проводится построение схемы);
  \item расположение элементов;
  \item расстановка связей между узлами;
  \item пользовательские программные описания функционирования узлов.
\end{itemize}

Соответственно, процесс загрузки не только должен восстанавливать созданную схему, но и вести контроль целостности данных: если некоторые модули отсутствуют в системе, пользователь должен уведомляться об этом.
%
%
\subsubsection{Визуальное проектирование схем}
Данная функция является ключевой в работе приложения.

Система должна предоставлять пользователю возможность разработки схем в режиме визуальной разработки: элементы схем и их свойства, связи между узлами, функциональные описания должны создаваться с помощью элементов простого графического интерфейса.
Управление настройками приложения также должно осуществляться с использованием графических меню.

Для решения прикладных задач пользователю предоставляется рабочая область, в границах которой возможно создание элементов, их взаимное размещение и перемещение.
С помощью инструментов задаются соединения между узлами.
За счет использования контекстных меню назначаются свойства элементов.
%
%
\subsubsection{Подключение пользовательских расширений}
Поскольку модули ядра предоставляют пользователю крайне ограниченные возможности по созданию схем, функциональность программы должна расширяться за счет специализированных файлов расширений.
Каждое такое расширение содержит описание отдельного элемента любой функциональной сложности~--- от простой логической функции до развитой подсистемы, решающей целую задачу.

Каждое расширение должно загружаться в систему в качестве инструмента, позволяющего разместить описываемый узел на схеме.

Механизм подключения расширений должен быть простым и прозрачным для конечного пользователя: оператор может должен иметь возможность самостоятельно выбрать требуемые расширения, а также с легкостью установить расширение из файла.

Описанные операции подключения и установки должны осуществляться из графического интерфейса.
%
%
\subsubsection{Проверка корректности синтаксиса создаваемых пользователем описаний функционирования схемы}
Использование в системе узлов, содержащих программные ошибки, недопустимо.
Поскольку со смысловыми ошибками (ошибками составления алгоритмов) невозможно бороться автоматизированно, задача проверки корректности сводится к проверке синтаксиса программного кода созданной программы.

Проверка корректности синтаксиса должна осуществляться в соответствие со стандартом VHDL93, поскольку он является последним и наиболее актуальным стандартом данного языка \cite{bibilo}.

Проверка синтаксиса должна сопровождаться предоставлением информации о строке и позиции нахождения ошибки, а в противном случае~--- сообщением о корректности синтаксиса.
%
%
\subsubsection{Подсветка синтаксиса для создаваемого пользователем VHDL-кода}
Для упрощения задачи написания программного кода на языке VHDL в имеющемся в системе редакторе исходного кода должна быть реализована функция подсветки программного кода.
Подобное разделение существенно упрощает ориентирование в исходном коде за счет дополнительного структурирования.

С помощью цветовых обозначений должны выделяться специфичные для языка конструкции синтаксиса: обозначения типов, имена функций, условные конструкции, операторы циклов.
%
%
\subsubsection{Генерация VHDL-кода, описывающего структуру и функционирование схемы}
Система должна обеспечивать преобразование схемы в программный код на языке VHDL.

Преобразование должно осуществляться в соответствие со структурной концепцией, то есть программный код, полученный в результате, должен описывать схему в соответствие с ее непосредственной структурой.
Узлам схемы сопоставляются сущности, которые впоследствии путем композиции объединяются в подсистемы.

Функционирование целой подсистемы, таким образом, определяется функционированием входящих в нее элементов и их связями, а не ее поведением.
Так реализуется структурная концепция, в противоположность поведенческой.

Полученный код должен быть валидным~--- то есть, синтаксически и семантически корректным.
Только в этом случае становится возможным его использование в сторонних системах и программных пакетах.
%
%
\subsubsection{Экспорт схемы в графическом формате}
Экспорт должен осуществляться в изображение в графическом формате PNG.
Выбор данного формата обуславливается полноценной работой с прозрачностью, отсутствием артефактов при сжатии и небольшим размером файла при наличии крупных графически однородных областей изображения.
Схемы, создаваемые в SVE, обладают всеми перечисленными чертами.

Экспортируемое изображение должно точно повторять созданную схему, с сохранением положения элементов, расположения связей, содержимого пометок и подписей.
%
%
\subsubsection{Многодокументный режим работы}
Система должна обеспечить возможность единовременного открытия нескольких документов.
%
%
%
\subsection{Требования к видам обеспечения} \label{sec:ware}

\subsubsection{Требования к математическому обеспечению} \label{sec:ware:math}
Необходимо разработать алгоритмы, решающие следующие задачи:
\begin{itemize}
  \item Размещение, перемещение и выравнивание элементов схемы по сетке.\\
  Алгоритм, который осуществляет добавление и перемещение графических представлений узлов схемы в рамках рабочей области с выравниванием по сетке.
  \item Соединение элементов схемы связями.\\
  Алгоритм, который просчитывает прорисовку соединений элементов по заданным концевым и промежуточным точкам. Для этого определяются координаты портов (входов и выходов элементов схемы), промежуточного узла и осуществляется прорисовка ломаной прямоугольной линии.
  \item Представление схемы в виде XML-дерева.\\
  Алгоритм, который динамически изменяет XML-дерево, описывающее текущую схему по мере добавления, удаления и изменения свойств узлов схемы и связей между ними.
  \item Преобразование XML-дерева в VHDL-представление.\\
  Алгоритм, который представляет созданную схему в виде кода на языке VHDL. Для этого для каждого узла схемы создается своя сущность, описываемая файлом расширения, задается алгоритм функционирования, а работа конечной схемы описывается на основе структурного подхода к разработке на VHDL.
  \item Экспорт схемы в виде изображения.\\
  Алгоритм, сохраняющий схему в графический файл формата PNG. Отображает расположение элементов, связей между ними, надписей и пометок на схеме.
  \item Работа буфера обмена.\\
  Алгоритм, который позволяет копировать и вырезать элементы, дублируя и перемещая их в схеме;
  \item История операций.\\
  Алгоритм, который реализует возможность отмены внесенных изменений до определенного шага. Заключается в реализации стека, работающего с XML-деревьями схемы.
\end{itemize}

Также необходимо использованы готовые алгоритмы:
\begin{itemize}
  \item Алгоритм вычисления манхэттенской длины (прямоугольной метрики)~--- используется для определения расстояния перемещения элементов при перетаскивании. Позволяет оптимально определять расстояния с учетом прямоугольного размещения элементов с фиксированными дистанциями.
  \item Алгоритмы чтения и записи XML-документов.
  \item Алгоритм поиска элементов в XML-дереве.\\
  Этот алгоритм позволяет выбирать из дерева документа узлы по заданному параметру~--- имени узла или уникальному идентификатору.
\end{itemize}
%
%
\subsubsection{Требования к информационному обеспечению} \label{sec:ware:info}
Необходимость использования баз данных отсутствует.

Ввод и вывод документов должен осуществляться с использованием файлов.
Форматы файлов необходимо разработать.
Также необходимо обеспечить экспорт данных в форматы .VHDL и .PNG.

Доступ к основным данным системы должен осуществляться в однопользовательском режиме: в единый момент времени только один пользователь может работать над одним проектом.
Однако, одни и те же файлы расширений могут одновременно использоваться несколькими пользователями (например, при запуске приложения с сетевого диска).

%
%
\subsubsection{Требования к программному обеспечению} \label{sec:ware:soft}
Работоспособность системы должна обеспечиваться для следующих программных платформ:
\begin{itemize}
  \item Microsoft Windows версии XP и выше;
  \item любой GNU/Linux-дистрибутив;
  \item Mac OS X версии 10.6 и старше.
\end{itemize}
Для запуска системы на указанных платформах установленные в системе, либо поставляемые вместе с запускаемыми файлами системы библиотеки Qt версии не ниже 5.1.

Для сборки приложения из исходных кодов необходимы:
\begin{itemize}
  \item библиотеки Qt версии не ниже 5.1;
  \item среда разработки Qt Creator соответствующей версии.
\end{itemize}

Для обеспечения функции проверки корректности синтаксиса пользовательских описаний схемы необходимо наличие либо установленного в системе, либо поставляемого в комплекте с запускаемыми файлами системы компилятора GHDL версии не ниже 0.29.
%
%
\subsubsection{Требования к техническому обеспечению} \label{sec:ware:hard}
Для обеспечения работоспособности и приемлемой скорости работы системы необходимы следующие совместимые компоненты ЭВМ:
\begin{itemize}
  \item центральный процессор с тактовой частотой не менее 2 Гц;
  \item оперативная память объемом не менее 1 ГБ;
  \item жесткий диск с не менее, чем 100 МБ свободного дискового пространства;
  \item видеокарта с не менее, чем 128 МБ видеопамяти;
  \item устройства ввода: мышь, клавиатура;
  \item устройства вывода: монитор с разрешением не менее 1024 на 768 точек.
\end{itemize}