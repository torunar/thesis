Процесс разработки SVE был достаточно трудоемок, была проделана значительная работа по разработке архитектуры системы, которая бы позволила не только решить поставленные задачи, но и разделить графический интерфейс и непосредственную работу с документом.
В результате подобного разделения было получено гибкое решение, которое может быть с легкостью встроено в сторонние продукты, а внесение изменений в интерфейс и ядро системы может проводиться независимо.

Значительные усилия были затрачены на проектирование структуры документов с целью, во-первых, обеспечения полноты и неизбыточности отражаемой в документе информации, во-вторых, возможности реализации простых и эффективных механизмов внесения изменений, в-третьих, исключения дополнительных преобразований для обработки информации.
Полученное решение позволяет выполнять рабочие операции и чтение и запись документа без промежуточных форматов~--- исключительно путем оперирования с XML-деревом.

С поставленными задачами система справляется в полной мере, реализуя все заявленные функции.

Хотя система создавалась для использования в учебном процессе, ее можно рассматривать как универсальное решение.
VHDL, выбранный в качестве языка экспорта, позволяет использовать результаты работы программы как в рассмотренных в пункте \ref{sec:characteristics:analogue} аналогичных САПР, так и в отдельных продуктах:
\begin{itemize}
  \item системах для синтеза схем по описанию;
  \item виртуальных измерительных приборах;
  \item симуляторах протекания процессов в системе;
  \item компиляторах программ для ПЗУ ПЛИС.
\end{itemize}

Созданная система имеет дальнейшие перспективы для развития и совершенствования.
Значительно облегчить процесс использования системы может разработка дополнительного компонента, нацеленного на конструирование пользовательских расширений --- сейчас создание расширений ведется в ручном режиме, что может быть затруднительно для пользователя.

К недостаткам системы можно отнести ограниченные возможности встроенного редактора программного кода, обеспечивающего лишь базовую подсветку синтаксиса языка VHDL.
Данный недостаток устаним за счет привлечения сторонних программных решений --- таковым, например, может стать компонент для редактирования исходных кодов <<Scintilla>>.

<<Scintilla>> предоставляет ряд возможностей:
\begin{itemize}
  \item Сворачивание структурных блоков текста (классов, функций, циклов).
  \item Автоматическая установка отступов.
  \item Подсветка парных или непарных (незакрытых) скобок.
  \item Автоматическое завершение используемых в файле имен типов, функций, переменных.
  \item Всплывающие подсказки о параметрах функций.
\end{itemize}

Использование <<Scintilla>> позволит добиться более совершенного отображения получаемого программного кода и превращения средства просмотра в мощный инструмент разработки.

Дальнейшую работу над системой планируется вести в следующем направлении:
\begin{enumerate}
  \item внедрить разработанную версию системы в учебный процесс;
  \item провести первое пользовательское тестирование с целью выявления пожеланий пользователей;
  \item выпустить новую версию системы с учетом результатов пользовательского тестирования;
  \item оптимизировать процесс получения программного описания;
  \item внести изменения в пользовательский интерфейс, расширяющий возможности проектирования схем.
\end{enumerate}