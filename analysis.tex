В пункте \ref{sec:characteristics:analogue} были рассмотрены некоторые САПР, являющиеся конкурентными по отношению к разрабатываемой системе, задействованные на сегодняшний день в учебном процессе.
В качестве таковых были выбраны <<Altera Quartus II>> и <<Xilinx ISE>> (<<Xilinx Designer Suite>>).

Причина подобного выбора заключаются в том, что компании <<Altera>> и <<Xilinx>> являются одними из крупнейших производителей ПЛИС в мире.
Они предоставляют конечным пользователям широкий спектр устройств~--- от простых и малопроизводительных ИС, подходящих для обучения и непрофессионального применения, до сложных комплексов, на основе которых могут быть построены полноценные производственные решения.

<<Altera Quartus II>>~--- программный комплекс, разработанный компанией <<Altera>>.
Он предоставляет возможности анализа и синтеза устройств на HDL, включая разработку их графических схем, выполнение временного анализа, разработку и анализ RTL-диаграмм, симуляцию и изучение реакции устройств на различные внешние воздействия, конфигурацию и прошивку конечного аппаратного комплекса.
Последняя версия~--- 13sp1, являющаяся сервисным расширением для предыдущей, 13-й, версии.

<<Altera Quartus II>> предоставляется в виде бесплатного для персонального использования и поддерживающего малый набор устройств веб-издания <<Quartus II Web Edition>>, обладающего ограниченной функциональностью и версии с платной подпиской <<Quatus II Subscription Edition>>.
Стоимость коммерческой лицензии~--- 2995\$ , она предоставляется на срок в 1 год, после чего должна быть продлена.
Стоимость продления лицензии~--- 2495\$.

<<Xilinx ISE>>~--- ПО, разработанное еще одним производителем ПЛИС, компанией <<Xilinx>>.
Оно во многом схоже с <<Altera Quartus II>>, выполняет аналогичные функции, как и <<Quartus II>> доступно в виде бесплатного веб-издания и платной версии с подпиской.
Последняя версия <<Xilinx ISE>>~--- 14.7~--- была выпущена в 2013 году.

И <<Altera Quartus II>>, и <<Xilinx ISE>> предоставляют следующие функции:
\begin{enumerate}
  \item Проектирование логических схем функционирования ПЛИС.\\
  Данная функция также носит название схемотехнического ввода.
  Она позволяет пользователю осуществлять проектирование ПЛИС в репрезентативной форме~--- с использованием элементов графического интерфейса путем расстановки компонентов на плате, указания их свойств и задания связей и механизмов взаимодействия между ними.
  Расставляться могут как простейшие логические элементы, так и сложные компоненты с заранее описанной функциональностью.
  Также могут размещаться порты и устройства ввода-вывода, а также аппаратные подсистемы, поддерживаемые конкретными моделями ПЛИС.
  \item Создание программного описания функционирования схем.\\
  Описание ведется с использованием интегрированной в САПР системы разработки.
  Данная ИСР предоставляет богатые возможности, упрощающие разработку ПО на поддерживаемых языках программирования (в основном~--- VHDL и Verilog).
  Компонентами таких ИСР являются, как минимум текстовый редактор, компилятор и отладчик.

  Текстовый редактор предоставляет некоторые возможности, упрощающие и ускоряющие написание и изменение кода, такие как подсветка синтаксиса, автодополнение, проверка правильности расстановки скобок, контекстная помощь по коду и многие другие.

  Компилятор осуществляет трансляцию программы, составленной на исходном языке высокого уровня, в эквивалентную программу на низкоуровневом языке, близком машинному коду.
  В контексте САПР ПО для ПЛИС он служит для получения двоичного файла прошивки для ПЛИС на основе созданного кода.

  Отладчик позволяет упростить процесс поиска возможных ошибок в коде: выполнять трассировку, отслеживать, устанавливать или изменять значения переменных в процессе выполнения кода, устанавливать и удалять контрольные точки или условия остановки.
  \item Статический временной анализ.\\
  Статический временной анализ~--- метод расчета временных параметров СБИС, не требующий полноценного электрического моделирования работы схемы.
  Время прохождения сигналов в реальной схеме в общем случае является вариативой величиной, значение которой невозможно определить точно при проектировании устройства: схема может выполнять разные операции, варьируется температура окружающей среды или напряжение, оно может изменяться под влиянием процесса производства.
  Нарушение порядка передачи сигналов может вести к ошибкам в работе схемы: данные будут ошибочно выставляться на линии передачи данных или сниматься с нее с нарушением временных последовательностей.
  Главной задачей статического временного анализа в этом случае становится проверка того, что во всех возможных вариациях сигнал прибудет по линиям связи на выход схемы в заданные временные рамки.
  Это задает главное условие безошибочной работы схемы.
  \item Загрузка программы в ПЗУ ПЛИС.\\
  В рамках данной функци осуществляется загрузка конфигурационной последовательности в ПЛИС FPGA и программирование ПЛИС CPLD и ППЗУ, то есть перенос программного описания на конкретное устройство.
  \item Автоматическое и ручное размещение и трассировка внутренних ресурсов ПЛИС в соответствии со списком цепей.\\
  Трассировкой называются правила размещения элементов и связей на кристалле ПЛИС.\\
  В процессе трассировки решаются сразу две задачи:
  \begin{itemize}[leftmargin=*]
    \item проверка реализуемости схемы на кристалле ПЛИС~--- если схема слишком сложна или неоптимально построена, она не может быть реализована на базе матрицы вентилей;
    \item оптимизация производительности~--- правильное размешение вентилей и уменьшение количества промежуточных связей между источником и приемником данных позволяет улучшить производительность системы на уровне прохождения сигналов и сложности обработки переносимых ими данных.
  \end{itemize}
  Средства автоматической трассировки, как правило, дают приемлимый результат для большинства схем, однако, полученная трассировка зачастую может быть улучшена~--- для этого предусмотрены средства ручной трассировки схемы.
  \item Создание диаграмм уровней регистровых передач (RTL).\\
  Диаграмма уровней регистровых передач -- описание работы синхронной цифровой схемы в терминах потоков сигналов (или пересылок данных) между аппаратными регистрами и логических операций над данными сигналами.
  Диаграмма RTL~--- высокоуровневое описание схемы, которое впоследствии преобразуется в низкоуровневое описание на различных HDL.
  \item Симуляция работы схемы.\\
  Разработчик может проверить функционирование работы полученной схемы без загрузки в ПЗУ устройства.
  САПР создаются производителями конкретных ПЛИС~--- это дает возможность симулировать работу именно на указанном устройстве с учетом его конструктивных особенностей.
\end{enumerate}

Структурно можно выделить следующие прикладные компоненты, входящие в состав САПР:
\begin{enumerate}
  \item Среда визуального проектирования~--- схемный редактор с возможностью расстановки элементов.
  \item Интегрированная среда разработки, объединяющая текстовый редактор, компилятор, отладчик и, как правило, средство прошивки.
  \item Средства симуляции, для задания входных данных, снятия выходных значений и визуализации сигналов.
  \item Вспомогательные средства~--- построитель RTL-диаграмм, средства трассировки и прошивки.
  \item Библиотеки компонентов, включающие как обширные логические модули и подсистемы для использования в схемах, так и описания возможностей особенностей функционирования конкретных моделей ПЛИС.
\end{enumerate}

Уже указывалось, что возможности данных САПР явно избыточны для решаемых в рамках учебного процесса задач, поэтому выделим наиболее актуальные функции для создаваемой системы:
\begin{itemize}
  \item проектирование логических схем функционирования ПЛИС;
  \item создание программного описания функционирования схем.
\end{itemize}

Подробная детализация каждой из этих функций позволяет получить конечные требования к функциям, выполняемым системой, указанным в пункте \ref{sec:functions}.

Поскольку список функций, выполняемых разрабатываемой САПР значительно уже такового конкурентных продуктов, изменению подвергается и список компонентов САПР.

В частности, полностью исключаются средства симуляции и вспомогательные средства, однако подобное облегчение архитектуры системы оправдано.
В результате работы <<SVE>> может быть получено описание схемы на VHDL~--- оно является универсальным средством передачи структуры устройства и его функционирования.
Полученный код может быть использован в большом количестве сторонних программ, начиная со средств визуализации и заканчивая многочисленными средствами симуляции, передающими функционирование разных устройств.

В случае необходимости, данные компоненты могут быть задействованы в разрабатываемой САПР, как сторонние компоненты, но на данном этапе разработки подобная задача не ставится в целях обеспечения функциональной полноты и неизбыточности ПО.

Таким образом, актуальными компонентами системы являются:
\begin{itemize}
  \item редактор схем для репрезентативного моделирования;
  \item редактор исходного кода для работы с VHDL;
  \item средство графического экспорта схемы;
  \item средство подключения расширений.
\end{itemize}

Последний компонент имеет особое значение в контексте разрабатываемой системы, являясь своеобразной заменой библиотеке компонентов специализированных САПР.